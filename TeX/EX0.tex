\documentclass[12pt,letterpaper,twoside]{article}

\usepackage{cme212}

\begin{document}

{\centering \textbf{Paper Exercise 0 -- C Warm-up\\ Due Friday, January 13, in class} \par}
\vspace*{-8pt}\noindent\rule{\linewidth}{1pt}

\paragraph{(0)} The poor ampersand (\&) and asterisk (*) get used in a lot of different ways in C++. Given the following declaration and initialization for {\tt a}, {\tt b}, and {\tt c}:
\begin{verbatim}
int a = 2, b = -3, c = -7;
\end{verbatim}
What can you say about the values of the following variables {\tt d} through {\tt h}:
\begin{compactenum}
\item[(a)] \texttt{int* d = \&c;}
\item[(b)] \texttt{int\& e = a;}
\item[(c)] \texttt{int f = b*c;} 
\item[(d)] \texttt{int g = *\&a;} 
\item[(e)] \texttt{int* h = a;}
\end{compactenum}

% % % % % % % % % % % % % % % % % % % % % % % % % % % % % % % % % % %

\paragraph{(1)} What is the output of the program below? Explain in your own words why.
\begin{cpp}
#include <iostream>

int f(int& a, int& b) {
  a = 3;
  b = 4;
  return a + b;
}

int main() {
  int a = 1;
  int b = 2;
  int c = f(a, a);
  std::cout << a << b << c << std::endl;
}
\end{cpp}

% % % % % % % % % % % % % % % % % % % % % % % % % % % % % % % % % % %

\paragraph{(2)} What is the output of the program below? Explain in your own words why.

\begin{cpp}
#include <iostream>

struct A
{
  A() : val(2) {}
  A(int v) : val(v+2) { v += 2; }
  A(A& a) : val(a.val+2) { a.val += 2; } 

  int val;
};

int main(int argc, char** argv)
{
  A a1;
  A a2(5);
  A a3(a1.val);
  A a4 = a3;
  
  std::cout << a1.val << a2.val << a3.val << a4.val << std::endl;
}
\end{cpp}

% % % % % % % % % % % % % % % % % % % % % % % % % % % % % % % % % % %

\end{document}
